% vim: spelllang=en spell textwidth=120
\documentclass[trojita]{subfiles}

\begin{document}

\chapter{Introduction}

\begin{quote}
  \itshape I also believe the only reason we still use email is that it’s impossible or very difficult to replace, kind
  of like Facebook in a way. \attrib{Heidar Bernhardsson \cite{email-difficult-to-replace-like-facebook}}
\end{quote}

The boom of mobile computing in the last few years has changed the way how people work with their e-mail.  After a short
period where it appeared that the webmail was the future and days of standard protocols were numbered, suddenly people
realized that in fact, there {\em is} a merit in having a standardized protocol for e-mail access.

Contemporary smartphones started to smear down the border between how a text message, an instant-messaging (IM)
communication and a traditional e-mail work.  Suddenly, it is common to have a single ``messaging'' application
aggregating communication history gathered from many distinct channels.

Even though some people prefer not to use these integrated facilities, their mere existence and certain demand from the
end-users present an unique opportunity for standardization --- it is easier to develop, test and deploy {\em one}
particular data provider suitable for use with many different services from multitude of vendors than having to deploy a
custom implementation for each social network which happens to be popular this year and on the continent the device
manufacturer decided to target.  Engineers working for cell phone and tablet vendors who are taking place in the IETF
standardization process are a clear evidence that the market recognizes this potential and that nobody wants to loose.

In this thesis, I would like to explore the IMAP protocol \cite{rfc3501} and its rich extension family, evaluating their
features by a prism of a {\em mobile client} --- a device which might have a decent amount of CPU and memory resources,
as common with today's smartphones and modern tablets, but whose network connection is prone to frequent interruptions
and might be payed based on the amount of transferred data, and whose battery would get extinguished in a few hours if
certain precautions are not taken.  It turns out that although many of the IMAP extensions are extremely useful for
increased efficiency of the client's operation, there are still quite a few opportunities for improvement, or outright
deficiencies to correct.

\section{Structure of the Thesis}

The next chapter (p. \pageref{sec:imap-protocol}) of this thesis provides a brief introduction to the baseline IMAP
protocol, its strengths and weaknesses and the general mode of operation.  The third chapter (p.
\pageref{sec:imap-extensions}) analyzes the existing IMAP extensions based on the ``layer'' on which they add features
to IMAP and on how they could be useful for the clients.  Their usefulness is illustrated through my experience in
development of {\em Trojitá}, an open source mobile IMAP e-mail client which I started and have been maintaining for the
last six years.  In chapter four (p.  \pageref{sec:proposed-extensions}), I have selected three completely different
areas in which the state of IMAP, as of 2012, can still be built upon.  My improvements are delivered in the format of
the so-called {\em Internet Drafts}, materials directly used by the IETF as the source of the RFC documents, the
specifications which drive innovation on the Internet.  All of the extensions which I propose were presented for expert
review through the relevant communication channels and are on track to become the Internet-Drafts and, hopefully, RFCs.

After the theoretical section, chapter five (p. \pageref{sec:mobile-imap}) evaluates the status of the extension support
among a selection of existing mobile clients available on the market and on real devices.  Chapter six serves as a short
introduction for programmers into how Trojitá, the mobile e-mail client developed for this thesis, operates.  The work
is concluded and a discussion of the work planned in future is available from chapter seven (p.
\pageref{sec:conclusion}).

All of the extensions which I am proposing are available in their traditional, formal format in Appendix \ref{sec:id-manuscripts}
(p. \pageref{sec:id-manuscripts}).

Finally, as Trojitá is a free software, open-source project, Appendix \ref{sec:acknowledgement} (p.
\pageref{sec:acknowledgement}) acknowledges the work of other developers who have contributed to Trojitá over the years
and lists the open source libraries required for its operation.  It also mentions two companies which decided to build
their e-mail products on top of Trojitá.  The instructions on how to build Trojitá from source as well as the structure
of the accompanying CD are in Appendix \ref{sec:source-cd} (p. \pageref{sec:source-cd}).

\end{document}
