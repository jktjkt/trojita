% vim: spelllang=en spell


\documentclass[12pt,notitlepage]{report}
%\pagestyle{headings}
\pagestyle{plain}

%\frenchspacing

\usepackage[utf8]{inputenc}
%\usepackage{czech}
\usepackage{a4wide}
%\usepackage{index} % nutno použít v případě tvorby rejstříku balíčkem makeindex
%\usepackage{fancybox} % umožňuje pokročilé rámečkování :-)
\usepackage{graphicx} % nezbytné pro standardní vkládání obrázků do dokumentu

\usepackage[left=4cm]{geometry}
\usepackage{xspace}

%\newindex{default}{idx}{ind}{Rejstřík} % zavádí rejstřík v případě použití balíku index

\title{Trojita: A Qt IMAP Client}
\author{Jan Kundrát}

\newcommand{\trojita}{Trojita\xspace}

\hyphenation{WebKit Imap Trojita}

%\date{}

\begin{document}


\begin{titlepage}
\begin{center}
%\ \\

\vspace{15mm}

\large
Charles University in Prague\\
Faculty of Mathematics and Physics\\

\vspace{5mm}

{\Large\bf BACHELOR THESIS}

\vspace{10mm}

\includegraphics[scale=0.3]{logo.eps}

\vspace{15mm}

%\normalsize
{\Large Jan Kundrát}\\
\vspace{5mm}
{\Large\bf IMAP E-mail Client}\\
\vspace{5mm}
Department of Software Engineering\\
%\end{center}
\vspace{15mm}

\large
\noindent Supervisor: Mgr. Vlastimil Babka
\vspace{1mm} 

\noindent Study Program: Computer Science, Programming

\vspace{20mm}

%\begin{center}
2009
\end{center}

\end{titlepage}

\normalsize
\setcounter{page}{2}
\ \vspace{10mm} 

\noindent I'd like to thank my supervisor, Mgr. Vlastimil Babka, for his numerous
advices during the writing of this thesis, and Ms. Anna Adamcová for her great
patience and support.

\vspace{\fill}
\noindent I hereby declare that I wrote this thesis myself using only the
referenced sources. I also agree with lending and publishing of this thesis.

\bigskip
\noindent Prague, May 29, 2009 \hspace{\fill}Jan Kundrát\\

\tableofcontents

\newpage

\noindent
Název práce: IMAP E-mail Client\\
Autor: Jan Kundrát\\
Katedra (ústav): Katedra softwarového inženýrství\\
Vedoucí bakalářské práce: Mgr. Vlastimil Babka\\
e-mail vedoucího: Vlastimil.Babka@mff.cuni.cz\\

\noindent Abstrakt:  V předložené práci studujeme ... Uvede se abstrakt v
rozsahu 80 až 200 slov. Lorem ipsum dolor sit amet, consectetuer adipiscing
elit. Ut sit amet sem. Mauris nec turpis ac sem mollis pretium. Suspendisse
neque massa, suscipit id, dictum in, porta at, quam. Nunc suscipit, pede vel
elementum pretium, nisl urna sodales velit, sit amet auctor elit quam id tellus.
Nullam sollicitudin. Donec hendrerit. Aliquam ac nibh. Vivamus mi. Sed felis.
Proin pretium elit in neque. Pellentesque at turpis. Maecenas convallis.
Vestibulum id lectus. Fusce dictum augue ut nibh. Etiam non urna nec mi mattis
volutpat. Curabitur in tortor at magna nonummy gravida. Mauris turpis quam,
volutpat quis, porttitor ut, condimentum sit amet, felis.\\

\noindent Klíčová slova: IMAP, e-mail, SMTP

\vspace{10mm}

\noindent
Title: IMAP E-mail Client\\
Author: Jan Kundrát\\
Department: Department of Software Engineering\\
Supervisor: Mgr. Vlastimil Babka\\
Supervisor's e-mail address: Vlastimil.Babka@mff.cuni.cz\\

\noindent Abstract: In the present work we study ... Uvede se anglický abstrakt
v rozsahu 80 až 200 slov. Lorem ipsum dolor sit amet, consectetuer adipiscing
elit. Ut sit amet sem. Mauris nec turpis ac sem mollis pretium. Suspendisse
neque massa, suscipit id, dictum in, porta at, quam. Nunc suscipit, pede vel
elementum pretium, nisl urna sodales velit, sit amet auctor elit quam id tellus.
Nullam sollicitudin. Donec hendrerit. Aliquam ac nibh. Vivamus mi. Sed felis.
Proin pretium elit in neque. Pellentesque at turpis. Maecenas convallis.
Vestibulum id lectus. Fusce dictum augue ut nibh. Etiam non urna nec mi mattis
volutpat. Curabitur in tortor at magna nonummy gravida. Mauris turpis quam,
volutpat quis, porttitor ut, condimentum sit amet, felis. \\

\noindent Keywords: IMAP, e-mail, SMTP

\newpage

\chapter{Introduction}

Hic sunt Leones.

\subsection{Motivation}

While there are certainly numerous stable and widely-used Mail User Agents on
the market, none of them fulfils my expectations of reasonable performance and
efficiency.  Indeed, most of these applications started as MUAs speaking the
POP3 protocol \footnote{The POP3 protocol\ldots FIXME}, adding support for IMAP
later on.  Typically, this support was added long since the initial design phase
of the product was completed, resulting in suboptimal design
choices.\cite{crispin-ten-commandments} Some clients, like the \ldots FIXME

\subsection{Structure of the thesis}

In the following chapter, we provide a gentle introduction to the world of
IMAP and related standards like MIME.  Following that, we move to explain some
design choices made during the development proces of \trojita, as well as
elaborate on the internal application architecture and structure.  User
documentation is the next chapter, and we finish the whole thesis by a wrap-up
providing an overview of what we implemented and how is the result usable.

\chapter{IMAP and Related Technologies}

Electronic mail, or a SMTP-mail, is a public service suitable for automated
message exchange among connected entities.  Interoperability is guaranteed by
several {\em Internet standards}, usually codified in the form of RFC documents.
Subject to these standards are various protocols specifying the rules of what
entities can communicate to each other as well as definition of the format of
all transmitted messages. In this section, we provide a gentle introduction to
the numerous standards which deal with this highly complex topic.

\section{Basic Concepts}

An {\em email message}, or a RFC-822\footnote{As defined by the RFC 822 standard
and subsequently modified by RFC 2822 \cite{rfc-2822} and others} message,
consists of three parts: {\em Envelope}, {\em Header} and {\em Body}. The {\em
envelope} is the only relevant part for mail exchange among Internet
hosts; it includes basic information like addresses of the sender and recipient.
A {\em header} provides more detailed metadata about the message, from
human-readable sender addresses and message route traces to user-defined
fields.  Some pieces of this information are used when the original delivery
fails for some reason, like the {\tt Return-Path} header.  The {\em header} also
serves as a basis for an {\em IMAP envelope data} which is explained later
in chapter \ref{imap-envelope}.  Finally, a {\em message body} is what an
average user typically refers to as ``the e-mail''.  It might contain just a
plain US-ASCII text, an HTML message with embedded images or it could be a
recursively defined entity with rich tree structure.

An {\em IMAP server} is a host in the Internet which provides access to local
mail store via the standard protocol, IMAP4rev1 in this case. The IMAP server
might be located in user's employer's data center or on the user's laptop, for
example.

An {\em MTA}\footnote{Mail Transfer Agent} or an {\em SMTP Server} is an
Internet host whose purpose is delivery of the RFC 2822 mail.  MTAs in the
Internet communicate with each other, determining routing information from
MX-records\footnote{Mail eXchange, a special record in the Domain Name Service
which specifies what servers are responsible for accepting all incoming e-mail
for the particular domain} in the DNS.  As a common practice nowadays, these
servers also accept outgoing e-mail from their own users, provided the
connection is properly authenticated, either by simple fact that it originates
from a trusted network or that the user has provided her credentials, typically as
a user name and password combination or in the form of an X.509 certificate.

A {\em Mailbox} or a {\em Mail Folder} is a place on the IMAP server where the
messages are logically stored.  A mailbox might contain number of child
mailboxes if the server implementation allows it.  Certain IMAP software has to
be told in advance whether the mailbox just being created is supposed to contain
only other mailboxes or only regular e-mail messages, while others allow free
mixing of both.

\section{IMAP-specific Attributes}

A message in the IMAP sense is {\em immutable}, that is, its contents (like
header, structure of the body or the actual message parts) can't change, ever.
Each message also has mutable attributes, the most important being {\em flags}.

Messages in a remote IMAP mailbox are identified by two partially independent
numbers, the {\em sequence number} and the {\em UID}.  Sequence numbers start at
one and are consecutive, that is, they dynamically change when a message in the
middle of the mailbox is removed.  This means that one particular message can
have multiple sequence numbers assigned over time and that one sequence number
can refer to totally unrelated messages, as the numbering changes.

On the other hand, the {\em UIDs} work differently.  When a message is delivered
to the mailbox, it receives a UID one greater than the previous one.  After the
UID is assigned, it will never change for that particular message, and the
mailbox remembers the biggest assigned UID so far.  This means that UIDs are
never recycled over the normal life span of a mailbox and they can be used as a
persistent identifier to one message in a particular mailbox.

Sometimes a non-standard situation might happen, e.g. when a third-party
accessed the mailbox data behind the back of the IMAP server.  Perhaps the user
tried a utility program that removes huge attachment from an otherwise valuable
message, or a spam classifier modified message headers directly, bypassing
standard mechanisms for proper flagging.  In any such case, the IMAP guarantee
of immutable nature of the messages doesn't hold, and a compliant IMAP server
has to communicate this problem to all clients.  The {\em UIDVALIDITY} mailbox
attribute serves this purpose.  Whenever the UIDVALIDITY changes, a compliant
client is required to throw away any cached information about a mailbox because
the UID assignment starts over again.

UID and UIDVALIDITY combined together form a 64bit integer which is absolutely
guaranteed to be an unique identifier for any message in the given mailbox -- if
the IMAP client once retrieved any immutable part from mailbox A with this 64bit
identifier, subsequent queries for the part identified by the same 64bit number
in the same mailbox can be satisfied from the client's cache.

It should be noted that this UID/UIDNEXT numbering is strictly per-mailbox and
not per-server, and that different users accessing the same server might see the
same 64bit number for totally unrelated messages.  While RFC3501 \cite{rfc-imap}
strongly recommends that UIDVALIDITY should be kept constant if at all possible,
a compliant IMAP client has to deal with UIDVALIDITY changes nonetheless.

The \label{imap-envelope}{\em IMAP envelope} is a data structure holding
interesting values determined from various message headers.  It is {\em not} the
same thing as the SMTP envelope, which typically stores less information.

The {\em Internal date} holds a time stamp about when the message was delivered
to the IMAP server.  {\em Size} attribute is the number of octets required for
storing a full copy of message in RFC2822 format.  {\em Body structure}
represents the tree-like structure of the MIME message.

The \label{imap-msg-flags}{\em IMAP flags} is an attribute holding a set of zero
or more named tokens associated with the message.  A flag can be either {\em
permanent} or {\em session-only}.  Permanent flags are stored in the mailbox
itself, while the session-only flags disappear on a subsequent reconnect to the
mailbox.  Some of the flags are defined by the RFC 3501 standard itself, others
are added by various recommendations and extension standards.  An IMAP server
might allow IMAP clients to store their own message flags.  A special means of
communicating this via the protocol exist\footnote{The {\tt PERMANENTFLAGS}
response code}.

Finally, message contents itself is distributed in form of {\em message texts}.
Clients are free to choose from retrieving MIME message parts separately or as
one blob.

\section{MIME}

Multipurpose Internet Mail Extensions \cite{rfc-2045} \cite{rfc-2046}
\cite{rfc-2047}, or just MIME, is a set of standards for extending the old SMTP
mail from 7bit ASCII texts to rich multimedia contents.  Parts of the standard
deal with encapsulation of non-ASCII characters in message headers, as well as
providing instructions about how to embed non-textual data in e-mail messages.

\subsection{Message as a Container}

An important aspect in MIME is its introduction of an internal structure to
e-mail message bodies.  Previously, a message body was just a plain text, while
from the codification of this standard, it became possible to exchange more
structured information.  Contemporary users are probably familiar with HTML
e-mails containing embedded images and a PDF attachment for seamless printing.
Users routinely send mails with attached photographs from their holidays, or an
archive containing business presentation.

MIME standard achieves this by defining a {\em content type} for message
bodies.  This content type, usually set by an RFC 2822 header of a
message, determines how a compliant mail client displays the message.  The
old-school non-MIME mails have a default MIME type of {\tt text/plain}, an
unformatted text in 7bit US-ASCII encoding.

Another portion of the standard defines generic-purpose containers, i.e.
abstract content types which contains other message parts within.  Using
these containers, a message can suddenly evolve from an unstructured blob of
text to a deep tree of unrelated body parts.

Most of these containers are defined in RFC 2045 \cite{rfc-2045} and 2046
\cite{rfc-2046}, one was added in RFC 2387 \cite{rfc-multipart-related}.

The {\tt multipart/alternative} encoding provides a mean of transmitting a
message that comes in several different formats, but each of them provides the same data.  The
MUA is expected to pick one of the underlying parts and display it, perhaps
allowing the user to override this selection as a nice bonus.

The {\tt multipart/mixed} is a generic catch-all MIME type for multipart
messages.  If a compliant MUA sees a multipart content type that it can not
recognize, it should behave as if it was the mixed multipart.  Typical action is
showing all embedded parts next to each other.

There are several more content types defined, but describing them in detail is
out of scope for this thesis.

\subsection{International Characters in Messages}

Original SMTP mail as defined by RFC 822 is suitable only for transferring
7bit data.  Nowadays, 8bit transport channels are far more common, both for representation of
international characters and transfer of binary data.  Therefore, the MIME
family defines {\em transfer encodings} for conversion of generic 8bit binary
data into a 7bit character stream and a portable way of expressing national
characters using 7bit ASCII characters only.

For the former, two standard encodings are available, the {\em Quoted printable}
and {\em Base64} (as well as a ``fake'' {\em binary} or {\em 8bit} encoding for
transport of binary data over less restrictive tunnels), the latter is supported
by a similar approach, as defined in RFC 2047 \cite{rfc-2047}.

\subsection{MIME Support in IMAP}

Much of the MIME work can be offloaded to the IMAP server.  This is especially
true for message structure parsing, where the IMAP specification provisions ways
to explore the ``tree'' of a message, as well as fetch the resulting body parts
separately (and even including byte ranges for more convenient download).

The IMAP protocol doesn't, however, handle the second part of MIME, nor much of
other header parsing.  While it is common to ask the server for just a named
subset of RFC 2822 mail headers, the returned data are not pre-parsed in any
way.  Therefore, much of the work from coalescing several header parts to
decoding international character data is left as an exercise to the client.

Fortunately, various message attributes in the IMAP protocol can serve as a
substitute for parsing the RFC 2822 headers.  Sender and recipient data is
available from the IMAP {\tt ENVELOPE}, message size as {\tt RFC822.SIZE}, IMAP
itself provides extensions for message threading \cite{rfc-threading} et cetera.

\section{IMAP Protocol Flow}

Upon a successful connection to a remote IMAP server, a client might choose to
authenticate itself if the server hasn't pre-authenticated it automatically
(perhaps as a result of being run by a wrapper script under non-privileged user
who can't switch to any other anyway).  After that or in case the
authentication is not required, the connection enters the {\em authenticated}
state.  In this phase, no mailbox is selected and only a subset of commands is
valid.  Clients can, for example, ask for listings of mailbox tree, get
quick information about number of messages in a particular mailbox or otherwise
manage mailboxes as whole.  They have to, however, {\em select} a mailbox in
order to do anything else, like retrieving messages or marking them as read.

A mailbox can be opened read-only or for both reading and writing, provided the
authenticated user has enough privileges.  Retrieving messages can be done in
both modes, but write operations (like storing a new message to mailbox or
manipulating the message flags) require read-write access.

\subsection{Mailbox Synchronization}

For efficiency reasons, each client usually keeps some data in its persistent
cache.  Good candidates for caching are message texts, a copy of body structure or
perhaps even flags from the previous session.  Because some server replies, namely
the {\tt EXPUNGE} which informs the client that a message has been removed from
the mailbox, refer to messages only by their sequence numbers and clients
usually identify messages by UIDs in order to allow message texts caching, a
compliant client has also to keep its UID-to-sequence number mapping up-to-speed
with the server.  Doing so for just a visible (in the GUI) part of the mailbox
might seem tempting, but the risks involved therein (like not knowing what to
purge from local cache as a result of received EXPUNGE without extra {\tt
FETCH}) are high and usually outweigh possible initial savings.

\chapter{\trojita Design}

\section{Overview}

\section{Model-View Architecture}

\section{Parser}

\section{Low-level Parser}

\section{Cache}

\chapter{Other IMAP Implementations}

\chapter{User's Guide}

\section{About Trojita}

\trojita is a standards-compliant IMAP Mail User Agent. Designed with portability
and efficiency in mind, it allows the user to operate over congested network
lines with high latency, while still providing sufficient comfort.

\section{Installation}

\subsection{Prerequisites}

In order to build \trojita from source, some libraries and helper programs are
required to be present on the system.

\begin{enumerate}
    \item{CMake 2.6}
    \item{Qt 4.4}
    \item{Working implementation of the TR1 C++ standard}
\end{enumerate}

Recommended version of the Qt framework \cite{qt} is at least 4.5. \trojita,
however, includes compatibility code for older versions of Qt. Minimal supported
version is Qt/4.4. Any older release can't be supported, as the WebKit HTML
engine which we use for e-mail rendering rendering was added during the 4.4
development. If users choose to build against 4.4, some \trojita features might
be missing.

Unlike most Qt projects, \trojita uses CMake \cite{cmake} instead of {\tt qmake},
especially due to author's familiarity with the former system.

\trojita makes use of some standard C++ features that were adding during the TR1
\cite{std-tr1} standardisation, namely the {\tt std::tr1::shared\_ptr} smart
pointer. A working implementation of said class is required.  \trojita build
system will make use of compiler's built-in TR1 support, if available, and
resort to using Boost if the compiler doesn't support this extension.

Some 3rd party code is also bundled in the source code distribution.  A list of
such products including the reason for not just using an external library
follows:

\begin{enumerate}
    \item{QwwSmtpClient \cite{qwwsmtpclient} is a Qt SMTP client library written
        by Witold Wysota.  Released to public only in mid May 2009, obtaining a
        package of it might prove challenging.  The copy in \trojita's codebase
        also contains some fixes which didn't made it to the publicly released
        version yet.}
    \item{ModelTest \cite{modeltest}, a class for verifying certain assumptions
        of the Interview Model-View framework of Qt.  Qt Software (former
        Trolltech) doesn't ship it with Qt sources, so \trojita includes a copy
        for developers' convenience.  This feature is not turned on by default
        in release builds.}
    \item{QtIconLoader \cite{qticonloader}, another helper library from Qt
        Software.  Purpose of it is providing a unified icon set, as per
        relevant XDG standards.}
    \item{KCodecs \cite{kcodecs}, an implementation of UTF-7, Base64,
        Quoted-Printable and related RFC 2047 standards.  These functions are
        reasonably self-contained, so they are bundled with \trojita instead of
        depending on quite heavyweight {\tt kdelibs}.}
\end{enumerate}

\subsection{Obtaining source code}

Source code is available from the CD attached to this thesis or online at FIXME.

\subsection{Installing}

Unpack the source tarball, prepare the build environment and launch the compile
process by entering these commands:

\begin{enumerate}
    \item{{\tt tar -xf trojita-1.0.tar.bz2}}
    \item{{\tt cd trojita-1.0}}
    \item{{\tt mkdir \_build}}
    \item{{\tt cd \_build}}
    \item{{\tt cmake -DCMAKE\_BUILD\_TYPE=RelWithDebInfo ..}} (the {\tt ..} are
        not a typo)
    \item{{\tt make}}
\end{enumerate}

After a while, a new binary, {\tt trojita}, will appear in the build directory.
Launch it to start \trojita.

\section{Using \trojita}
\subsection{Configuration}

All configuration is, thanks to the Qt's QSettings, stored in target platform's
common storage location. The only supported mean of configuring \trojita is
through its built-in configuration dialog that will automatically appear on the
first run.  It can also be launched any time later from the application menu.

% FIXME: screenshot

On the first page, user is asked to provide identity data. These are used for
composing e-mail messages.

Second page is the most important one in configuration of \trojita. It is the
place where the user tells the application which IMAP server is it supposed to
talk to, which credentials should be used for authentication et cetera.

One unusual thing which is not common among other MUAs is the choice of how to
connect to the IMAP server.  Most users are probably familiar with connections
over TCP, optionally secured by SSL/TLS, but the last option, the {\em Local
Process}, is unique to \trojita. If selected, it replaces the usual TCP
connection to the IMAP server with launching a local process, which is in turn
used instead of the remote IMAP server. The most interesting use case is to set
up SSH connection to the remote server (along with proper SSH keys) and use that
as a secure, reliable and convenient way for accessing e-mail, without having to
enter passwords all the time. For example, to launch an instance of Dovecot IMAP
server on host {\tt mail.example.org}, user should enter {\tt ssh
mail.example.org dovecot --exec-mail imap}.

The last page serves for configuring the message composition and sending.
\trojita supports mail submission via SMTP\footnote{Simple Mail Transfer
Protocol} and local {\tt sendmail}-compatible MSA\footnote{Mail Submission
Agent\cite{rfc-msa}, a local gateway whose purpose is forwarding e-mail to the
network of MTAs} application.

\subsection{Basic usage}

After the configuration is complete, the user is presented with a familiar user
interface. The left pane presents a view of remote mailboxes, or folders, on the
IMAP server. The top widget contains list of messages in currently opened
folder, and finally the bottom area is reserved for displaying of the actual
e-mail messages.

Depending on the network parameters and server load, it might take a while
before the list of mailboxes appears. The delay is only significant when
launching \trojita for the first time, for the retrieved list of mailboxes is
cached on a persistent disk cache for later use.

After the list of mailboxes is loaded, user can click on them to open them in
the {\em Message List} widget on the bottom right corner of the application
window.  After having selected a mailbox, the list of messages appears shortly.
Again, the loading is faster next time \trojita is launched.

Depending on user settings, \trojita typically won't transfer all messages
metadata upon opening a mailbox.  This feature is implemented to conserve
both server's and client's resources and to be nice to the network traffic.  In
a typical mode, that is, when not being told to be extra-mean with resources,
\trojita fetches only metadata of those messages that are displayed on the
visible screen region, and some more to make scrolling smoother.  More on this
topic is said in another chapter of this thesis.

Message contents is transfered on-demand only, that is, when user selects a
message for viewing.  As before, all persistent contents once downloaded from
the server is cached in a persistent on-disk cache for future use.  This applies
to both message metadata and the real contents of e-mails.

\subsection{Managing Messages and Mailboxes}

Messages displayed to user are automatically marked as read after some delay.
The idea behind this feature is to allow users to peek on a particular message,
shortly judge its contents, classify it as a ``process later'' and then move on
to another one.  This is a common approach shared with other MUAs as well.

Manipulating with {\em message flags} (see chapter \ref{imap-msg-flags}) is possible
via the context menu (appearing on the message list) as well by standard
keyboard shortcuts.  The {\tt M} key toggles the {\em Read} status and {\tt
Delete} key schedules messages for removal.

In consistency with usual IMAP approach (and unlike the common modus operandi on
mainstream desktop MUAs), \trojita has no concept of {\em Trash}
\footnote{A {\em Trash} is a mailbox that presents a view of all deleted
messages.  It might or might not be a real mailbox on the server, or just a
virtual folder implemented purely in the e-mail client.} for
messages.  Instead, each message can be marked as deleted.  If flagged as such,
\trojita displays a graphical indicator next to the message subject.  A deleted
message won't be physically removed from the mailbox until the user explicitly
requests such an action.  This operation, called {\em expunge} in the IMAP
dictionary, is available from the application menu.  Expunging happens on one
mailbox at a time, there is no way to tell the IMAP server to permanently remove
all deleted messages from all mailboxes.

To create new mailbox, user is expected to use the context menu on the left {\em
mailbox view} pane.  Due to technical limitations, some IMAP servers distinguish
among folders that can contain other mailboxes and folders for message store
only.  Additionally, there is no reliable way for the IMAP client to tell
whether the remote server is affected by this limitation or not.  Therefore,
\trojita uses the same approach as other common MUAs and ask user what kind of
mailbox she wants to create.

% FIXME screenshot

Mailbox removal is triggered from the context menu as well.  Due to the possibly
serious consequences of this action (there is no way to undelete a removed
mailbox), user is presented with a confirmation dialogue asking her whether she
is serious in her intent to remove the mailbox.

Messages can be copied or moved between mailboxes by simple drag-and-drop
operations.  To place a copy of one or more messages into another mailbox,
simply select the messages to copy in the opened mailbox and then drag them with
mouse to the target mailbox.  If user wants to move them instead of performing a
copy, she is supposed to hold the {\tt Shift} key.  This behavior is determined
by the Qt framework, so it might vary among the supported platforms.

\subsection{Message Composition}

Message Composition widget is triggered by the corresponding item in the menus.
The displayed dialog should be familiar to most users.

\subsection{Checking Mail}

\trojita automatically checks the current folder for new messages.  If the
remote server supports the IDLE extension \cite{rfc-idle}, the method used is
extremely friendly to both network and server resources.  In cases the server
doesn't support this feature, \trojita resorts to polling with all it
disadvantages.

\section{Advanced usage}

\subsection{Partial Message Fetching}

Thanks to unique features of the IMAP protocol, it isn't necessary to fetch the
whole message in order to display just a fragment of it.  \trojita handles this
automatically, fetching message parts on demand.  It also won't transfer message
metadata (like its RFC2822 envelope) unless really needed.

\subsection{Offline Mode}

When put into {\em offline mode}, \trojita will prevent itself from accessing
the network at all.  Users will be still able to view already transfered data, but
no checks for new messages will be performed and those that aren't stored in
local offline cache won't have actual message bodies displayed until the user
selects either {\em online} or {\em expensive network} mode from the menu.

\subsection{Expensive Network}

Some users connect to their IMAP servers over a slow or otherwise unsuitable
network.  Examples include dial-up connections or various mobile solutions like
GPRS\footnote{General packet radio service (GPRS) is a mobile data service
typically offering low bandwidth, high latency Internet connection with billing
based on the amount of transfered data} where the overall convenience of having
all messages immediately available is usually held back by the need to conserve
network resources.  In such environments, it might be desirable to fetch only
the absolute minimum data required to provide properly synchronized view of the
remote mailbox.

In this situation, user is expected to choose the {\em Expensive Network} mode
from the application menu.  \trojita will not preload data to improve
interactivity, nor will it show all message parts by default.  Upon viewing a
message, only parts smaller than a certain size are transfered automatically,
others will be loaded after clicking a button.  This on-demand loading is
implemented separately for each part, allowing great flexibility for user to
choose what is interesting and what can be ignored.

\subsection{Online Mode}

{\em Online mode}, the common mode of operation, is what users are familiar from
other IMAP clients.  \trojita employs intelligent preloading of interesting
contents to minimize waiting times, yet doesn't blindly download everything,
including parts that won't be ever shown.

\subsection{Local Cache}

All data once transfered form the network\footnote{Subject to some restrictions
like maximal message part size and cache parameters} are kept in a persistent
offline on-disk cache.  If a message data is transfered twice, it means that at
least one of the following conditions was met:

\begin{enumerate}
    \item{The server couldn't guarantee the immutability of messages.  This might
        happen when some process touches on-disk data internal to the IMAP
        server in a non-standard way.}
    \item{User has changed her settings.  Whenever any data used to access the
        remote server changes, the only safe thing to do is to throw away
        anything in the local cache and start over again.}
\end{enumerate}

\subsection{Full Tree View}

As an optional part of the GUI, the {\em full tree view}, activated by the
corresponding item in the menu, provides a full view of the server contents,
including mailbox hierarchy, messages in mailboxes and mailbox parts.  This
widget is a nice demonstration of a complex nature of MIME messages -- while
most users are probably familiar with the concept of attachments, the truth is
that MIME defines a fully recursive structure that one message can consist of.
For example, it's perfectly possible to have a message composed of six parts,
each of them being a forwarded message, and each of these forwarded messages
being rich-structured as well.  \trojita does its best at formatting the message
in a sane and familiar way, but unfortunately some senders don't really follow
the same philosophy and produce messages whose structure leaves much to be
desired.  More on this matter can be found in another place in this thesis
\ref{FIXME}.

\chapter{Conclusion}

\trojita, the IMAP e-mail client implemented for this thesis,\ldots

\begin{thebibliography}{99}
\addcontentsline{toc}{chapter}{Bibliography}
    \bibitem{rfc-imap}Crispin, M.: Internet Message Access Protocol - Version
        4rev1, {\em RFC3501}, March 2003
    \bibitem{rfc-idle}Leiba, B.: IMAP4 IDLE command, {\em RFC2177}, June 1997
    \bibitem{crispin-ten-commandments}Crispin, M.: Ten Commandments of How to
        Write an IMAP client, September 2006, Washington,
        {\tt http://www.washington.edu/imap/documentation/commndmt.txt.html}
    \bibitem{client-best-practices}Sirainen, T., Cridland, D. et al.: Best
        Practices for Implementing an IMAP Client, {\tt
        http://www.imapwiki.org/ClientImplementation}
    \bibitem{rfc-2822}Resnick, P. (editor) et al.: Internet Message Format, {\em
        RFC2822}, April 2001
    \bibitem{rfc-msa}Gellens, R., Klensin, J.: Message Submission, {\em
        RFC2476}, December 1998
    \bibitem{rfc-2045}Freed, N., Borenstein, N.: Multipurpose Internet
        Mail Extensions (MIME) Part One: Format of Internet Message Bodies, {\em
        RFC2045}, November 1996
    \bibitem{rfc-2046}Freed, N., Borenstein, N..: Multipurpose Internet
        Mail Extensions (MIME) Part Two: Media Types, {\em RFC2046}, November 1996
    \bibitem{rfc-2047}Moore, K.: MIME (Multipurpose Internet Mail Extensions)
        Part Three: Message Header Extensions for Non-ASCII Text, {\em RFC2047},
        November 1996
    \bibitem{rfc-multipart-related}Levinson, E.: The MIME Multipart/Related
        Content-type, {\em RFC 2387}, August 1998
    \bibitem{rfc-content-disposition}Troost, R., Dorner, S., Moore, K.:
        Communicating Presentation Information in Internet Messages: The
        Content-Disposition Header Field, {\em RFC2183}, August 1997
    \bibitem{rfc-threading}Crispin, M., Murchison, K.: Internet Message Access
        Protocol - SORT and THREAD Extensions, {\em RFC5256}, June 2008
    \bibitem{qt} Qt Reference Documentation, {\tt http://doc.trolltech.com/4.5/}
    \bibitem{cmake} CMake Documentation, {\tt
        http://www.cmake.org/cmake/help/documentation.html}
    \bibitem{std-tr1} ISO/IEC TR 19768, C++ Library Extensions
    \bibitem{qwwsmtpclient} Wysota, W.: QwwSmtpClient released, May 2009, \\
        {\tt
        http://blog.wysota.eu.org/index.php/2009/05/13/qwwsmtpclient-released/}
    \bibitem{modeltest} Qt Software, ModelTest, February 2008, {\tt
        http://labs.trolltech.com/page/Projects/Itemview/Modeltest}
    \bibitem{qticonloader} Qt Software, February 2009, \\ {\tt
        http://labs.trolltech.com/blogs/2009/02/13/freedesktop-icons-in-qt/}
    \bibitem{kcodecs} The KDE Project: kdelibs, {\tt
        http://api.kde.org/4.x-api/kdelibs-apidocs/}
\end{thebibliography}

\end{document}
