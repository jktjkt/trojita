% vim: spelllang=en spell


\documentclass[12pt,notitlepage]{report}
%\pagestyle{headings}
\pagestyle{plain}

%\frenchspacing

\usepackage[utf8]{inputenc}
%\usepackage{czech}
\usepackage{a4wide}
%\usepackage{index} % nutno použít v případě tvorby rejstříku balíčkem makeindex
%\usepackage{fancybox} % umožňuje pokročilé rámečkování :-)
\usepackage{graphicx} % nezbytné pro standardní vkládání obrázků do dokumentu

\usepackage[left=4cm]{geometry}
\usepackage{xspace}

%\newindex{default}{idx}{ind}{Rejstřík} % zavádí rejstřík v případě použití balíku index

\title{Trojita: A Qt IMAP Client}
\author{Jan Kundrát}

\newcommand{\trojita}{Trojita\xspace}

%\date{}

\begin{document}

%\csprimeson % zapne jednoduché psaní českých uvozovek pomocí klasických znaků, ale potom pozor 
             % na originální apostrofy, které budou chybně interpretovány!!!

%%% Následuje první, úvodní, strana bakalářské práce. Jednotlivé položky nahraďte dle vlastních
%%% údajů. Změnit podle konkrétní délky jednotlivých položek můžete i zalomení řádků.
\begin{titlepage}
\begin{center}
%\ \\

\vspace{15mm}

\large
Charles University in Prague\\
Faculty of Mathematics and Physics\\

\vspace{5mm}

{\Large\bf BACHELOR THESIS}

\vspace{10mm}

%\includegraphics[scale=0.3]{logo.eps} FIXME

\vspace{15mm}

%\normalsize
{\Large Jan Kundrát}\\
\vspace{5mm}
{\Large\bf IMAP E-mail Client}\\
\vspace{5mm}
Department of Software Engineering\\
%\end{center}
\vspace{15mm}

\large
\noindent Supervisor: Mgr. Vlastimil babka
%\hskip20mm   je-li odlišné od uvedeného názvu katedry či ústavu 
\vspace{1mm} 

\noindent Study Program: Computer Science, Programming
%\hskip20mm oboru (směru),  příp. název studijního plánu

\vspace{20mm}

%\begin{center}
2009
\end{center}

\end{titlepage}

\normalsize
\setcounter{page}{2}
\ \vspace{10mm} 

\noindent I'd like to thank my supervisor, Mgr. Vlastimil Babka, for his numerous
advices during the writing of this thesis, and Ms. Anna Adamcová for her great
patience.

\vspace{\fill}
\noindent I hereby declare that I wrote this thesis using only the referenced
sources. I also agree with lending and publishing of this thesis.

\bigskip
\noindent Prague, May 29, 2009 \hspace{\fill}Jan Kundrát\\

% FIXME SIGNATURE!!!

\tableofcontents

\newpage

\noindent
Název práce: IMAP E-mail Client\\
Autor: Jan Kundrát\\
Katedra (ústav): Katedra softwarového inženýrství\\
Vedoucí bakalářské práce: Mgr. Vlastimil Babka\\
e-mail vedoucího: Vlastimil.Babka@mff.cuni.cz\\

\noindent Abstrakt:  V předložené práci studujeme ... Uvede se abstrakt v rozsahu 80 až 200 slov. Lorem ipsum dolor sit amet, consectetuer adipiscing elit. Ut sit amet sem. Mauris nec turpis ac sem mollis pretium. Suspendisse neque massa, suscipit id, dictum in, porta at, quam. Nunc suscipit, pede vel elementum pretium, nisl urna sodales velit, sit amet auctor elit quam id tellus. Nullam sollicitudin. Donec hendrerit. Aliquam ac nibh. Vivamus mi. Sed felis. Proin pretium elit in neque. Pellentesque at turpis. Maecenas convallis. Vestibulum id lectus. Fusce dictum augue ut nibh. Etiam non urna nec mi mattis volutpat. Curabitur in tortor at magna nonummy gravida. Mauris turpis quam, volutpat quis, porttitor ut, condimentum sit amet, felis.\\

\noindent Klíčová slova: IMAP, e-mail, SMTP

\vspace{10mm}

\noindent
Title: IMAP E-mail Client\\
Author: Jan Kundrát\\
Department: Department of Software Engineering\\
Supervisor: Mgr. Vlastimil Babka\\
Supervisor's e-mail address: Vlastimil.Babka@mff.cuni.cz\\

\noindent Abstract: In the present work we study ... Uvede se anglický abstrakt v rozsahu 80 až 200 slov. Lorem ipsum dolor sit amet, consectetuer adipiscing elit. Ut sit amet sem. Mauris nec turpis ac sem mollis pretium. Suspendisse neque massa, suscipit id, dictum in, porta at, quam. Nunc suscipit, pede vel elementum pretium, nisl urna sodales velit, sit amet auctor elit quam id tellus. Nullam sollicitudin. Donec hendrerit. Aliquam ac nibh. Vivamus mi. Sed felis. Proin pretium elit in neque. Pellentesque at turpis. Maecenas convallis. Vestibulum id lectus. Fusce dictum augue ut nibh. Etiam non urna nec mi mattis volutpat. Curabitur in tortor at magna nonummy gravida. Mauris turpis quam, volutpat quis, porttitor ut, condimentum sit amet, felis. \\

\noindent Keywords: IMAP, e-mail, SMTP

\newpage


\chapter{User's Guide}

\section{About Trojita}

\trojita is a standards-compliant IMAP Mail User Agent. Designed with portability
and efficiency in mind, it allows the user to operate over congested network
lines with high latency, while still providing sufficient comfort.

\section{Installation}

\subsection{Prerequisites}

In order to build \trojita from source, some libraries and helper programs are
required to be present on the system.

\begin{enumerate}
    \item{\texttt{cmake 2.6}}
    \item{Qt 4.4}
    \item{Working implementation of the TR1 C++ standard}
\end{enumerate}

Recommended version of the Qt framework is at least 4.5. \trojita, however,
includes compatibility code for older versions of Qt up to 4.4. Any older
version can't be supported, as the WebKit HTML engine was added during the 4.4
development. If users choose to build against 4.4, some \trojita features might
be missing.

Unlike most Qt projects (and due to author's familiarity with this particular
build tool), \trojita uses CMake instead of {\tt qmake}.

\trojita makes use of some standard C++ features that were adding during the TR1
standardisation, namely the {\tt std::tr1::shared\_ptr} smart pointer. A working
implementation of said container is required.  \trojita build system will make
use of built-in TR1 support, if available, and resort to using Boost if the
compiler doesn't support this extension.

\subsection{Obtaining source code}

Source code is available from the CD attached to this thesis or online at FIXME.

\subsection{Installing}

Unpack the source tarball, prepare the build environment and launch the compile
process by entering these commands:

\begin{enumerate}
    \item{{\tt tar -xf trojita-1.0.tar.bz2}}
    \item{{\tt cd trojita-1.0}}
    \item{{\tt mkdir \_build}}
    \item{{\tt cd \_build}}
    \item{{\tt cmake -DCMAKE\_BUILD\_TYPE=RelWithDebInfo ..}}
    \item{{\tt make}}
\end{enumerate}

After a while, a new binary, {\tt trojita}, will appear in the build directory.
Launch it to start \trojita.

\section{Using \trojita}
\subsection{Configuration}

All configuration is, thanks to the Qt's QSettings, stored in a usual space on
the target platform. The only supported mean of configuring \trojita is through
its built-in configuration dialog that will automatically appear on the first
run.  It can also be launched it any time later from the menu.

% FIXME: screenshot

On the first page, user is asked to provide identity data. These are used for
composing e-mail messages.

Second page is the most important one in configuration of \trojita. It is the
place where the user tells the application which IMAP server is it supposed to
talk to, which credentials should be used for authentication et cetera.

One unusual thing which is not common among other MUAs is the choice of how to
connect to the IMAP server.  Most users are probably familiar with connections
over TCP, optionally secured by SSL/TLS, but the last option, the {\em Local
Process}, is unique to \trojita. If selected, it replaces the usual TCP
connection to the IMAP server with launching a local process, which is in turn
used instead of the remote IMAP server. The most interesting use case is to set
up SSH connection to the remote server (along with proper SSH keys) and use that
as a secure, reliable and convenient way for accessing e-mail, without having to
enter passwords all the time.

The last page serves for configuring the message composition and sending.

\subsection{basic usage}

After the configuration is complete, the user is presented with a familiar user
interface. The left pane presents a view of remote mailboxes, or folders, on the
IMAP server. The top widget contains list of messages in currently opened
folder, and finally the bottom area is reserved for displaying of the actual
e-mail messages.

Depending on the network parameters and server load, it might take a while
before the list of mailboxes appears. The delay is only significant when
launching \trojita for the first time, for the retrieved list of mailboxes is
cached on a persistent disk cache for later use.

After the list of mailboxes is loaded, user can click on them to open them in
the {\em Message List} widget on the bottom right corner of the application.
After having selected a mailbox, the list of messages appears shortly. Again,
the loading is faster next time \trojita is launched.

Depending on user settings, \trojita typically won't transfer all messages
metadata upon opening a mailbox.  This feature is implemented to conserve
both server's and client's resources and to be nice to the network traffic.  In
a typical mode, that is, when not being told to be extra-mean with resources,
\trojita fetches only metadata of those messages that are displayed on the
visible screen region, and some more to make scrolling smoother.  More on this
topic is said in another chapter of this thesis.

Message contents is transfered on-demand only, that is, when user selects a
message for viewing.  As before, all persistent contents once downloaded from
the server is cached in a persistent on-disk cache for future use.  This applies
to both message metadata and the real contents of e-mails.

\subsection{Managing Messages and Mailboxes}

Messages displayed to user are automatically marked as read after some delay.
The idea behind this feature is to allow users to peek on a particular message,
shortly judge its contents, classify it as a ``process later'' and then move on
to another one.  This is a common approach shared with other MUAs as well.

Manipulating with {\em message flags} \ref{imap-attr-message-flags} is possible
via the context menu (appearing on the message list) as well by standard
keyboard shortcuts.  The {\tt M} key toggles the {\em Read} status and {\tt
Delete} key schedules messages for removal.

In consistency with usual IMAP approach (and unlike the common modus operandi on
mainstream desktop MUAs), \trojita has no concept of {\em Trash}
\footnote{A {\em Trash} is a mailbox that presents a view of all deleted
messages.  It might or might not be a real mailbox on the server, or just a
virtual folder implemented purely in the e-mail client.} for
messages.  Instead, each message can be marked as deleted.  If flagged as such,
\trojita displays a graphical indicator next to the message subject.  A deleted
message won't be physically removed from the mailbox until the user explicitly
requests such an action.  This operation, called {\em expunge} in the IMAP
dictionary, is available from the application menu.  Expunging happens on one
mailbox at a time, there is no way to tell the IMAP server to permanently remove
all deleted messages from all mailboxes.

To create new mailbox, user is expected to use the context menu on the left {\em
mailbox view} pane.  Due to technical limitations, some IMAP servers distinguish
among folders that can contain other mailboxes and folders for message store
only.  Additionally, there is no reliable way for the IMAP client to tell
whether the remote server is affected by this limitation or not.  Therefore,
\trojita uses the same approach as other common MUAs and ask user what kind of
mailbox she wants to create.

% FIXME screenshot

Mailbox removal is triggered from the context menu as well.  Due to the possibly
serious consequences of this action (there is no way to undelete a removed
mailbox), user is presented with a confirmation dialogue asking her whether she
is serious in her intent to remove the mailbox.

Messages can be copied or moved between mailboxes by simple drag-and-drop
operations.  To place a copy of one or more messages into another mailbox,
simply select the messages to copy in the opened mailbox and then drag them with
mouse to the target mailbox.  If user wants to move them instead of performing a
copy, she is supposed to hold the {\tt Shift} key.  This behavior is determined
by the Qt framework, so it might vary among the supported platforms.

\subsection{Message Composition}

Message Composition widget is triggered by the corresponding item in the menus.
The displayed dialog should be familiar to most users.

\subsection{Checking Mail}

\trojita automatically checks the current folder for new messages.  If the
remote server supports the IDLE extension \cite{rfc-idle}, the method used is
extremely friendly to both network and server resources.  In cases the server
doesn't support this feature, \trojita resorts to polling with all it
disadvantages.

\section{Advanced usage}

\subsection{Offline Mode}



\begin{thebibliography}{99}
\addcontentsline{toc}{chapter}{Bibliography}
    \bibitem{rfc-idle}Leiba B.: IMAP4 IDLE command, {\em RFC2177}


 \bibitem{abraham-marsden}Abraham R., Marsden J. E.: {\em Foundations of Mechanics}, Addison-Wesley, Reading, 1985.
 \bibitem{derbes}Derbes D.: {\em Reinventing the wheel: Hodographic solutions to the Kepler problems}, Am. J. Phys. {\bf 69} (2001) 481--489.
 \bibitem{kvasnica}Kvasnica J.: {\em Teorie elektromagnetického pole}, Academia, Praha, 1985.
\end{thebibliography}

\end{document}
