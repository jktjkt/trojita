% vim: spelllang=en spell textwidth=120
\documentclass[trojita]{subfiles}

\begin{document}

\chapter{Trojitá's Architecture}

This chapter provides a brief introduction to the architecture of Trojitá from a programmer's point of view.

\section{Overview of Components}
\todo[inline]{Talk about the internal structure of Trojita --- the Model, Tasks, proxy models}

Trojitá makes heavy use of certain idioms common in Qt programming and in object-oriented software development in
general.

At the highest layer lies to GUI, graphical code managing the actual user interaction.  This code contains no knowledge
of IMAP or any other e-mail protocol; it is simply a graphical presentation layer specific to the desktop version of
Trojitá.  In the releases intended for mobile platforms, the traditional {\tt QWidget}-based GUI is replaced by a
variant built on top of Qt's QML \cite{qml}, a framework especially suited for touch-centric design.

Any interaction with IMAP is initiated through the model-view framework \cite{qt-mvc}.  A core class encapsulating a
representation of a single IMAP server, the {\tt Model} class, is accompanied by various proxy models and other
utilities to segregate and transform the data into better shape suitable for the upper layers.

Any action which shall take effect is, however, not performed by any of the model-related classes.  Trojitá utilizes the
concept of {\em tasks}, a set of single-purpose classes each serving a distinct role.  Examples of such tasks are
``obtain a connection'', ``synchronize a mailbox'', ``download a message'' or ``update message flags''.

One layer below the Tasks, the Parser is located.  This class along with its support tools converts data between a
byte stream from/to the network and higher-level commands and responses which are utilized by the upper
layers.  Actual network I/O operations are handled through a thin wrapper around Qt's own {\tt QIODevice} called {\tt
Streams}.~\footnote{The {\tt QIODevice} is wrapped to allow for transparent compression using the deflate algorithm.  Due
to the historic reasons, the {\tt Stream} subclasses use has-a over the is-a approach; this was required back when
Trojitá shipped IO implementations not based on the {\tt QIODevice} class.}

\subsection{Handling Input/Output}

\subsection{The Concept of Tasks}

The Tasks are designed to collaborate with each other, and there is a network of dependencies on how they can be used,
so that \ldots

\subsection{Models and Proxies}

\section{The Mobile Version}
\todo[inline]{Also mention the version usable on cell phones}

\section{Regression Testing}
\todo[inline]{Time for a shameless plug about the test suite}
\todo[inline]{Mention the low-level optimizations --- string deduplication, results of profiling,\ldots}


\end{document}
