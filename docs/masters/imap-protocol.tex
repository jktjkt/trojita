% vim: spelllang=en spell textwidth=120
\documentclass[trojita]{subfiles}

\begin{document}

\chapter{IMAP Protocol Introduction}
\label{sec:intro}

\begin{abstract}
This chapter provides a gentle introduction to peculiarities of the IMAP protocol.  We also present detailed analysis of
the available extensions and how they can be used to improve the overall IMAP experience.
\end{abstract}

\section{IMAP}

The IMAP protocol, as defined by RFC~3501~\cite{rfc-imap}, is an internet protocol suitable for managing e-mail folders
and individual messages stored on a remote mail server.  In contrast to the older POP3 protocol~\cite{rfc-pop3}, IMAP is
really intended to serve as an {\em access} protocol.  Where a POP3 client would happily download a full message from
the mail server, store it into a local mailbox and perform all further processing locally, the IMAP mode of operation is
much more complicated.  These complications, however, bring a whole slew of new features and interesting applications
along.

For one, IMAP presents a single, authoritative place storing messages -- that feature alone is a must in today's world
where people are expecting to be able to access their mail from their cell phones.  Furthermore, given that all messages
are located on a single place, it is possible to perform efficient, server-side operations over the whole mailstore,
like searching or sorting.  IMAP also makes it possible to access individual message parts like attachments separately,
eliminating the need to download a huge message before reading a short accompanying textual information.  Finally,
advanced servers can recognize clients with limited resources and only present a subset of messages to them.

At the same time, IMAP is an old protocol burdened by many compatibility warts.  Its designers were struggling with
people objecting to novel ideas due to legacy code in their mail implementations.  Over the years, though, various
protocol extensions appeared.  Some of them are extremely useful for contemporary clients, yet they cannot be relied on
because there is no general agreement on what extensions are really crucial, and hence available in most of IMAP
servers.

The rest of this chapter provides a quick overview of the basic IMAP concepts and how they relate to the usual client's
workflow.

\section{Basic Features}

An IMAP server exports a set of {\em mailboxes}, folders which contain individual messages (and further mailboxes, if
the server allows that).  Each message can be identified either by its {\em sequence number}, an order in which it
appears in mailbox, or by its {\em UID}.  Sequence numbers are by definition very volatile (deleting the first message
in a mailbox changes sequence numbers of all subsequent messages) while the UIDs provide better chances of persistence
across reconnects.~\footnote{It shall be noted that IMAP does {\em not} guarantee UIDs to be persistent at all.  The
reason behind this decision was to allow IMAP to publish messages from obsolete mail stores which could not have been
extended to support UIDs at all.  Even today, UID changes have to be expected when signalled through {\tt UIDVALIDITY}.}
When the UIDs have to be invalidated for some reason, a per-mailbox integer property {\tt UIDVALIDITY} shall be
incremented to signal its clients that previously used UIDs are no longer valid.

\subsection{Cache Filing Protocol}

As Mark Crispin, the principal author of the IMAP standard, has to say~\cite{crispin-imap-cache-filing}, IMAP is a {\em
cache filing protocol}.  That means that whatever the server thinks about a mailbox state is {\em the truth}, and any
state stored on the clients can be invalidated by the server at any time.  This critical design choice has impact on all
further operations.  IMAP clients which do not anticipate such a behavior~\footnote{Such clients are usually called
``POP3 clients converted to speak IMAP'' on various IMAP-related mailing
lists.~\cite{shannon-imap-clients-glorified-imap}} are bound to operate in an inefficient manner or fail in unexpected
scenarios.

The first issue which typically comes up on the {\tt imap-protocol} mailing list is treating UIDs as a persistent
identifier of some kind.  In fact, IMAP guarantees that a triple of (mailbox name, {\tt UIDVALIDITY}, {\tt UID}) will
not refer to {\em any other} message at no time, but there's no guarantee that the vary same message, quite possibly in
the same mailbox, will not get another UID in future.~\footnote{People have been trying to solve this issue for quite
some time, but no standardized solution is ready yet.  The recent iterations of these proposals concentrate on providing
a cryptographic hash of a message body, but is far from clear whether doing so would get any traction.  Furthermore, the
hashes are typically too long to serve as the only identifier of a message, so UIDs will definitely be around in
future.}  That said, the UIDs should not get invalidated too often and the IMAP protocol doesn't offer anything else, so
they are widely used (along with the UID and when limited to the scope of a single mailbox) as a semi-persistent
identification of a message.


\subsection{Caching Data}



% crispin-imap-cache-filing
% http://mailman2.u.washington.edu/pipermail/imap-protocol/2010-June/001144.html
% http://mailman2.u.washington.edu/pipermail/imap-protocol/2010-June/001156.html

% shannon-imap-clients-glorified-pop
% http://mailman2.u.washington.edu/pipermail/imap-protocol/2006-September/000261.html

\end{document}
