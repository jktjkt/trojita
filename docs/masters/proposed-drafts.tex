% vim: spelllang=en spell textwidth=120
\documentclass[trojita]{subfiles}

\begin{document}

\chapter{Proposed Extensions}

Previous chapters have shed some light on the complicated world of IMAP and showed how the protocol limitations affects
the users' experience.  I have also introduced some of the existing extensions which aim to address many shortcomings.
There are still quite a few issues which make lives of the client implementors harder than necessary, though.  At this
point, I present three extensions which fix race conditions, improve the effectiveness of the protocol or add new
features which contribute to smoother operation of the e-mail clients.  This broad range of changes was selected to
illustrate that improving IMAP can happen on many different levels.

Internet Drafts are usually prepared in a special system \cite{rfc-formatting} which handles the required strict
document formatting using ASCII art.  This chapter is therefore purposely very short, providing only the minimal
descriptions of the proposed extensions.  The Internet Drafts themselves are found in appendix \ref{sec:id-manuscripts}
on page \pageref{sec:id-manuscripts} and are an integral part of this thesis.

\section{Announcing UIDs of Newly Arriving Messages in QRESYNC: the ARRIVED Extension}
\label{sec:draft-arrived}

The first extension I have implemented addresses a race condition in the {\tt QRESYNC} extension \cite{rfc5162}.  In
{\tt QRESYNC}, the offset-based {\tt EXPUNGE} responses known from the baseline IMAP protocol are replaced by {\tt
VANISHED} responses which use UIDs.  Unfortunately, because the {\tt EXISTS} still informs about the number of new
deliveries only, without including the UIDs, and due to the fact that the IMAP server is explicitly
allowed~\footnote{{\em``Note that a VANISHED response caused by EXPUNGE, UID EXPUNGE, or messages expunged in other
connections SHOULD only contain UIDs for messages expunged since the last VANISHED/EXPUNGE response sent for the
currently opened mailbox or since the mailbox was opened.  That is, servers SHOULD NOT send UIDs for previously expunged
messages, unless explicitly requested to do so by the UID FETCH (VANISHED) command.''

``Note that client implementors must take care to properly decrement the number of messages in the mailbox even if
a server violates this last SHOULD or repeats the same UID multiple times in the returned UID set.  In general, this
means that a client using this extension should either avoid using message numbers entirely, or have a complete mapping
of UIDs to message sequence numbers for the selected mailbox.''}~\cite[p. 12]{rfc5162} --- in the
RFC language, {\em SHOULD} means that implementations are suggested to use the recommended behavior, but can deviate
from that as {\em ``there may exist valid reasons in particular circumstances to ignore a particular item''}
\cite{rfc2092}.} to include non-existing UIDs in the {\tt VANISHED} responses, a race condition exists where client does
not know about the full value of the sequence $\rightarrow$ UID mapping, which in turn violates RFC 5162's requirement
on clients having {\em ``a complete mapping of UIDs to message sequence numbers for the selected mailbox''}.

The proposed extension addresses this issue through the {\tt ARRIVED} response.  At the same time, it improves the
protocol efficiency by freeing the clients from a requirement to explicitly ask for message UIDs when a new message is
delivered.

In absence of the {\tt ARRIVED} extension, clients are required to perform an explicit UID rediscovery.  Servers which
already do not send non-existing UIDs in the {\tt VANISHED} responses will still benefit from implementing the {\tt
ARRIVED} response as the clients will not have to perform explicit {\tt UID SEARCH} operations on them upon new
deliveries.

Full text of the proposed extension in the format of an Internet-Draft suitable for IETF submission is included in
section \secref{sec:draft-imap-qresync-arrived}.

\section{Improving Incremental Threading through Modified INTHREAD}
\label{sec:draft-inthread-ext}
\todo[inline]{Incremental threading}

\section{Submitting Internet Mail --- the SENDMAIL Extension}
\label{sec:draft-sendmail}
\todo[inline]{Submission over IMAP}

POSTADDRESS might not work well with Sieve (and other server-side filters),
http://www.ietf.org/mail-archive/web/imapext/current/msg00828.html

\end{document}
