% vim: spelllang=en spell textwidth=120
\documentclass[trojita]{subfiles}

\begin{document}

\chapter{The Mobile IMAP}
\label{sec:mobile-imap}

Many of the existing IMAP extensions discussed in \secref{sec:imap-extensions} have the potential of improving the
client's operation tremendously.  At the same time, experience has shown that there is a certain chicken-and-egg problem
with new proposals where server vendors are not willing to invest their time into promising extensions which no client
supports yet and clients are not interested in implementing extensions which they cannot test for interoperability.  In
this chapter, I am trying to provide a concise summary of individual merit of these extensions.

\section{The Lemonade Profile}
\label{sec:lemonade-comparison}

The Lemonade profile, as defined in RFC 4550 \cite{rfc4550} in 2006 and later updated through RFC 5550 \cite{rfc5550}
during 2009, provides a list of extensions considered ``critical'' for any mobile IMAP e-mail client.  The set of
mandatory extensions is rather big, though, and to the best of my knowledge, there is {\em no} server on the market
implementing all of the compulsory features.  One might therefore wonder what were the reasons for this lack of general
availability of the Lemonade extension family.

\subsection{Cross-Service Requirements}

One unique feature of Lemonade is the possibility to {\em forward messages without their prior download}.  The three
ESMTP~\cite{rfc5321} and IMAP extensions, often referred to as the {\em Lemonade trio}, namely the {\tt CATENATE}, {\tt
URLAUTH} and {\tt BURL}, allow the clients to compose a message using existing parts available from the IMAP mail store,
provide a way of generating single-purpose ``pawn tickets'' for making the composed messages available to the submission
server, and replacing the {\tt DATA} SMTP command with a way of downloading the message from the IMAP server,
respectively.  This feature prevents having to transfer potentially huge data over the network three times --- once when
the users wants to read it, second time when the message is saved to the sent folder, and finally when delivering via
SMTP.

Unfortunately, a big problem with said approach is the fact that it mandates collaboration across different services ---
an explicit trust path between the IMAP and ESMTP servers have to be set up, which is a process prone to errors
\cite{qmf-fastmail-burl-bug}.  This matter is also complicated by the fact that no open source MTA~\footnote{Mail
Transfer Agent, typically an SMTP or ESMTP server} ships with official support for {\tt BURL}.~\footnote{Unofficial
patches exists for Postfix dating back to 2010 \cite{apple-postfix-burl}, but they have not been integrated into the
mainline version as of July 2012 (the {\tt postfix-2.10-20120715.tar.gz} development snapshot.}  Situation is better on
the IMAP server front with Cyrus supporting the {\tt URLAUTH} and {\tt CATENATE} extensions out-of-box with Dovecot's
support scheduled for its upcoming 2.2 release \cite{imap-server-extension-matrix}.

\subsection{Complicated Extensions}

Some of the extensions whose support is mandated by the Lemonade proposal seems to be notoriously hard for the server
vendors to implement.

A perfect example is the {\tt CONTEXT=SORT} extension \cite{rfc5267}.  As a client developer, I recognize its extreme
usefulness and appreciate its design.  Availability of such an extension would make it extremely easy to implement
live-updated sorting in my Trojitá (and Trojitá {\em does} make use of the sort context extension).  That said, given
that no IMAP server which I am aware of announces its availability, clients have to deal with the status quo in the
meanwhile.

The {\tt CONVERT} extension \cite{rfc5259} belongs to a similar category --- the features it offers, like the
server-side downscaling of JPEG images, would be {\em very} handy on a cell phone, yet no IMAP server known to the
author includes that functionality.

Both of these RFCs were published four and five years ago, respectively, and were designed by engineers working for an
IMAP server vendor.  One cannot therefore dismiss them altogether as a product of people not having any say in the
server development.  My opinion is that the allocation of engineering resources required for shipping a particular
feature in a finished product is based on another criteria than the research activity.

\section{State of Other Client Implementations}

To obtain a better understanding on how the existing solutions available on the market today use IMAP, this section
takes a look at some of the most popular solutions.

\subsection{Apple iOS}

Apple's devices generally ship with a decent implementation of their IMAP stack, an evaluation shared by independent
researchers \cite{isode-iphone4}.  The list of extensions supported by the application includes {\tt CONDSTORE} and {\tt
ESEARCH} for improved mailbox synchronization, {\tt COMPRESS} for transparent deflate compression and {\tt BURL} for the
forward-without download.

It is, however, surprising that their support of extensions aimed at making mailbox resynchronization more efficient
does not include the {\tt QRESYNC} extension \cite{rfc5162}, especially given that its implementation does not impose
much in terms of additional requirements on top the already-supported {\tt CONDSTORE}.

The iOS also notably does not use the {\tt IDLE} command at all.  The reason, according to a message allegedly sent by
Steve Jobs \cite{jobs-ios-idle}, is that is is {\em ``a power hungry standard''}.  Systematic measurements
\cite{wcdma-energy} \cite{cridland-fach-dch-measurements} and experience alike~\footnote{Mark Crispin's famous {\em ``I
have built on-demand networks which shut down until signaled back on, and then happily resumed all the active TCP
sessions even though the "network connection" had been powered off for days.''} \cite{crispin-no-ifup}.} shows, however,
that the mere act of having a TCP connection open with an occasional keep alive ``pings'' being transfered have no
significant impact on battery life on other platforms.

\subsection{Android's Native E-mail Client}

There is not much to be said about Android's native client's IMAP performance --- the stock client does not offer push
notification through {\tt IDLE} \cite{android-idle} and the list of extension identifiers referenced from the
application's source~\footnote{File \path{src/com/android/email/mail/store/ImapConnection.java} from the
\path{platform/packages/apps/Email} repository as of the {\tt android-4.1.1\_r3-35-g01c55fd} revision.} only references
the {\tt NAMESPACE}, {\tt UIDPLUS} and {\tt STARTTLS} capabilities.  None of the extensions which try to improve
synchronization performance (the {\tt ESEARCH}, {\tt CONDSTORE} and {\tt QRESYNC}) are available.  No provisions for
Lemonade's family are present at all.

Further analysis shows that the code is blocking and incapable of issuing or processing requests in parallel.  These
observations are consistent with what users generally describe as a ``slow'' experience.  This might not come surprising
given that Google would likely prefer its users to choose Google's own e-mail offering, the GMail, over various private
IMAP accounts for business reasons.

\subsection{Android's K-9 Mail}

The Android's K-9 mail is a fork of the original e-mail application from Google.  The developers have managed to add
support for two extensions, namely the {\tt IDLE} and the {\tt COMPRESS=DEFLATE}.  More advanced features like the {\tt
ESEARCH}/{\tt CONDSTORE}/{\tt QRESYNC} are however still missing.

Furthermore, comments like {\em ``TODO Need to start keeping track of UIDVALIDITY''}~\footnote{File
\path{src/com/fsck/k9/mail/store/ImapStore.java}, line 103 as of the Git revision {\tt 3.512-1249-g5ce0e19} of the K-9's
source code repository.} might make one feel nervous about the safety of the data being accessed.

\subsection{Nokia's Qt Messaging Framework}

\todo[inline]{Analyze the QMF}

\subsection{Trojitá}

\todo[inline]{Analyze Trojita}

\section{Evaluating Extensions}
\todo[inline]{Evaluate how useful/hard to implement they are}

\end{document}
